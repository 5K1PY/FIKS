\documentclass{article}
\usepackage[utf8]{inputenc}
\usepackage{amsmath}
\usepackage{amssymb}
\usepackage{graphicx}
\usepackage[czech]{babel}
\graphicspath{ {./images/} }

\title{3 Ryby}
\author{Daniel Skýpala}
\date{21. ledna 2022}

\begin{document}
\maketitle

\section*{Popis algoritmu}

Budueme provádět známý postup zametání roviny. Vezmeme si přímku, která je rovnoběžná s vektorem \([s_x, s_y]\) a protíná osu \(y\) v bodě
\([0, -\infty]\) a budeme ji pomalu posouvat do \([0, \infty]\), čímž vyzkoušíme všechny kombinace.

Nejdřív si rozmyslíme, že není zapotřebí zkoušet veškeré možné souřadnice, ale jenom ty, kde naše přímka protíná jeden z vrcholů našich zadaných mnohoúhelníků.
Je tomu tak proto, že vždy můžeme naši přímku posunout do jednoho z vrcholů, aniž bychom něco změnili. Mezi těmito vrcholy se totiž nic nemění.

Pro vrcholy každého mnohoúhelníka si spočítám, kde přímka rovnoběžná s \([s_x, s_y]\) procházející příslušným vrcholem protíná osu x. Můžu si všimnout,
že přímka bude současný mnohoúhelník protínat pouze když bude mezi přímkami, která protínají vrcholy mnohoúhelníka a jejich průsečík s osou \(y\)
(jeho druhá souřadnice) je co nejmenší / největší:
\begin{figure}[h]
    \centering
    \includegraphics{Intersects.png}
\end{figure}

Tím pádem jsou relevantní z každého mnohoúhelníka jen tyto 2 body.

Krátká vsuvka: Jak počítat průsečík s osou \(y\)? Aby byla naše přímka rovnoběžná s \([s_x, s_y]\) musí mít \(a = \frac{s_y}{s_x}\) (\(a\) jenom říká 
o kolik se změní \(y\) za \(\Delta x=1\)). Pokud chceme, aby přímka procházela daným bodem, souřadnice již známe, takže jen vyjádříme \(b\):
\[y = ax + b\]
\[y = \frac{s_y}{s_x}x + b\]
\[b = y - \frac{s_y}{s_x}x\]
Přičemž průsečík s osou \(y\) je \(f(0) = 0a+b = b\), takže jsme ho spočítali. Ale abychom se nemuseli potýkat se zlomky:
\[s_x b = s_x y - s_y x\]
A můžeme porovnávat hondotu \(s_x b\). Vzhledem k tomu, že hodnota \(s_x\) je kladná a stejná pro všechny vrcholy, tak platí:
\[b_1 < b_2 \Leftrightarrow s_x b_1 < s_x b_2\]
\[b_1 = b_2 \Leftrightarrow s_x b_1 = s_x b_2\]
\[b_1 > b_2 \Leftrightarrow s_x b_1 > s_x b_2\]
Čímž jsme se kompletně zbavili desetiných čísel.

Teď když víme, jak spočítat průsečíky s osou \(y\), můžeme si z každého mnohoúhelníka vzít vrchol s nejmenším a největším průsečíkem. K těm
s nejmenším průsečíkem si zapíšu, že jsou vstupní, těm největším výstupní.

Potom je všechny seřadím vzestupně. (Pokud mají dva body stejný průsečík s osou \(y\), tak nejdřív budu dávat ty vstupní, protože když přímka
protíná mnohoúhelník ve vrcholu, je to taky průsečík.) Potom je všechny projedu. Budu si udržovat nejvyšší počet protnutých mnohoúhelníků
a aktuální počet protnutých mnohoúhelníků a pro každý vrchol v seřazeném pořadí:
\begin{itemize}
    \item Pokud je to vstupní vrchol, od teď protínám o mnohoúhelník víc. Přičtu si aktuálnímu počtu 1.
    \item Pokud je to výstupní vrchol, protínám o mnohoúhelník míň. Odečtu si od aktuálního počtu 1.
    \item Pokud aktuální počet je vyšší jak maximum, uložím si ho jako maximum.
\end{itemize}
Na konci jen vypíšu maximum.

Celý tento postup funguje proto, že vyzkouším všechny možnosti, kam přímku můžu umístit.

\section*{Časová složitost}
Děláme následující operace (\(N\) - \#mnohúhelníku, \(B\) - \#bodů):
\begin{enumerate}
    \item Načtení vrcholů a mnohoúhelníků: \(O(B+N)\)
    \item Spočtení \(s_x b\) pro body a vybrání dvou pro každý mnohoúhelník: \(O(B)\)
    \item Setřízení: \(O(N \log N)\)
    \item Projití \(O(N)\)
\end{enumerate}
Celkem \(O(B + N \log N)\)

\section*{Paměťová složitost}
Pro každý bod si pamatujeme jeho hodnotu \(s_x b\). Potom si pamatujeme seřazený seznam. Celkem \(B + N\).

\end{document}