\documentclass{article}
\usepackage[utf8]{inputenc}
\usepackage{amsmath}
\usepackage{amssymb}
\usepackage{graphicx}
\usepackage{listings}
\usepackage{verbatim}
\graphicspath{ {./images/} }

\title{Klece}
\author{Daniel Skýpala}
\date{4. prosince 2021}

\begin{document}
\maketitle

\section{Zamyšlení}
Každé zvíře bude určitě v nějakém pavilonu. Takže nic nepokazíme na tom, že umístěme zvíře s nejmenší dravostí ze všech zvířat do pavilonu.
Dále pro podmínku \(|d_j − d_k| \leq p.\) si můžeme rozmyslet, že zvíře v pavilonu má nejmenší možnou dravost,
takže podmínku můžeme přepsat na:
\[d_j - d_{min} \leq p\]
(Pro každé \(d_j\) platí \(d_j \geq d_{min}\), takže levá strana je vždy kladná.) \\
Každé další zvíře, které do pavilonu přidáme, musí splňovat tuto podmínku, a tedy náležet intervalu:
\[[d_{min}, d_{min}+p]\]
Přidat větší zvíře nemůžeme, protože by sežralo \(d_{min}\) a menší zvíře než \(d_{min}\) ani neexistuje. \\
Takže přidáme do pavilonu všechny zvířata z tohoto intervalu, tím jsme část (alespoň 1) zvířat vyřadili a zbyla nám nová množina zvířat.
A na tu můžeme tento postup opakovat.

\section{Popis algoritmu}
Abychom byli schopni rychle zvířata hledat, tak si seřadíme podle dravostí (a uložíme si všechny indexy). Nastavíme \(i=0\) a poté opakujeme:
\begin{enumerate}
    \item Zvíře na pozici \(i\) je nejmenší nezařazené, přidáme ho do pavilonu. Zvýšíme \(i\) o 1.
    \item Dokud zvíře na pozici \(i\) může jít být spolu s nejmenším zvířetem v pavilonu:
        \begin{itemize}
            \item Zařadíme ho tam a zvýšíme \(i\) o 1.
        \end{itemize}
    \item Když současné zvíře se do pavilonu nevleze, vypíšeme indexy zvířat v pavilonu. (Pokud to má být stejně, jako v ukázokovém řešení, můžeme
    je i řadit). Dále začneme znovu od bodu 1.
\end{enumerate}
Skončíme, když nám dojdou zvířata (\(i\) po zvýšení bude větší rovno délce seznamu).

\section{Časová složitost}
Jednotlivé části:
\begin{itemize}
    \item Počáteční seřazení stihneme v \(O(N \log N)\).
    \item Poté pro každé zvíře (z \(O(N)\)), buď pro něj pavilon založíme a přidáme ho tam \(O(1)\), nebo ho přidáme do již stávajícího pavilonu \(O(1)\). Celkem \(O(N)\).
    \item Každý pavilon vypíšeme (v seřazené podobě složitost \(O(p \log p)\)):
    \[O\left(\sum_{i=0}^m p_i \log p_i\right) \in O(N \log N)\]
    (Pokud \(N = \sum_{i=0}^m p_i\).)    
\end{itemize}

Dáme-li to dohromady \(O(N \log N + N + N \log N) = O(N \log N)\).

\section{Paměťová složitost}
Pamatujeme si akorát seznam seřazených zvířat s indexy \(O(N)\) a zvířata v jednom pavilonu \(O(N)\). Celkem \(O(N)\).

\section{Kód}
\begin{verbatim}
n = int(input())
animals = list(map(int, input().split()))
p = int(input())
i = 0

animals = list(sorted(enumerate(animals), key=lambda x: x[1]))

while i < len(animals):
    min_animal = animals[i]
    pavilion = [min_animal[0]]
    i += 1
    while i < len(animals) and animals[i][1] <= min_animal[1] + p:
        pavilion.append(animals[i][0])
        i += 1
    print(f"({' '.join(map(lambda x: str(x+1), sorted(pavilion)))})")
\end{verbatim}

\end{document}