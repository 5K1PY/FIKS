\documentclass{article}
\usepackage[utf8]{inputenc}
\usepackage{amsmath}
\usepackage{amssymb}
\usepackage{graphicx}
\graphicspath{ {./images/} }

\title{Prohlídkové okruhy}
\author{Daniel Skýpala}
\date{16. prosince 2021}

\begin{document}
\maketitle
\hyphenation{thatshouldnot}
\section{Zamyšlení}

``Je možné předpokládat, že počet podmnožin křižovatek, kde z každé křižovatky jedné
podmnožiny je možné se dostat do všech ostatních křižovatek této podmnožiny a zpět, je tři.''

Můžeme si všimnout, že definice podmonožin křižovatek je ekvivalentní s definicí silně souvilé komponenty.
Každý vrchol leží v jedné právě jedné silně souvislé komponentě. Pokud hledáme cestu mezi vrcholy \(a, b\)
pojďme rozebrat jednotlivé případy, jestli existuje cesta mezi nimi v závislosti na tom,
v jakých jsou komponentách:

\begin{enumerate}
    \item Pro libovolné dva vrcholy ve stejné silné komponentě cesta existuje (z definice).
    
    \item A pokud jsou dva vrcholy v různých komponentách, tak musí existovat posloupnost hran mezi těmito dvěma vrcholy.
    V jedné komponentě se umíme dostat z každého vrcholu do každého, takže nás zajímají jen hrany mezi komponentami.
    Musí tedy existovat posloupnost hran mezi komponentami, taková že se dostaneme ze startovní komponenty do cílové.
    
\end{enumerate}
\section{Popis algoritmu}
Samotný alogritmus má tři části:
\begin{enumerate}
    \item Najití silně souvislývh komponent
    \item Spočítaní dosažizelnosti mezi komponentami
    \item Zodpovězení dotazů
\end{enumerate}

\subsection{Najití silně souvislých komponent}
Na určení silně souvislých komponent existuje více algoritmů, z čehož jsem si já vybral Korasajův.

Algoritmus probíhá následovně:
\begin{enumerate}
    \item Uděláme si inverzi \(G^T\) grafu \(G\) (pokud hrana v \(G\) vedla z \(a\) do \(b\), tak v \(G^T\) vede z \(b\) do \(a\)).
    \item Vyrobíme si zásobník.
    \item Opakovaně spouštíme prohledávání do hloubky na \(G^T\) z každého nenavštíveného vrcholu (nenavštivujeme již navštívené).
    Pokaždé když vrchol opouštíme, přidáme si ho do zásobníku.
    \item Postupně odstraňujeme vrcholy ze zásobníku a spouštíme na ně prohledávání do hloubky v \(G\) (a zase nenavštivujeme již navštívené).
    Všechny vrcholy, které navštívíme v jednom spuštění dfs, jsou v jedné komponentě.
\end{enumerate}

Proč toto vlastně funguje? Předtím než se pustíme do objasňování složitějších částí, vyřešeme ty jednoduší:
\begin{enumerate}
    \item \textit{V zásobníku bude každý vrchol právě jednou.} Každý vrchol tam určitě je, protože jsme ho navštívili (jinak bychom na něj spustili dfs)
    a tedy jsme ho museli i opustit (a při tom přidat do zásobníku). Zároveň, aby byl v zásobníku víckrát než jednou, museli bychom ho opustit,
    a tedy i navštívit víckrát. Nicméně my nenavštěvujeme vrcholy opakovaně.
    \item \textit{Každý vrchol je v právě jedné komponentě.} Na všechny nenavštívené vrcholy spouštíme dfs, pro každý vrchol v zásobníku (to jsou všechny vrcholy).
    No a každý navštívený vrchol je v jedné komponentě. Navštívené vrcholy nenavštěvujeme znovu.
    \item \textit{Pokud existuje cesta z A do B i z B do A, jsou v jedné komponentě, jinak jsou v různých komponentách.} Pojďme rozebrat všechny případy:
    \begin{itemize}
        \item \(A \not\leftrightarrow B\) Pokud neexistuje cesta ani jedním směrem, tak dfs v \(G\) nemá šanci dojít ve stejném spuštění
        z \(A\) do \(B\) a tedy je zařadit do stejné komponenty.
        \item \(A \rightarrow B\) V zásobníku bude \(B\) po \(A\). Buď jsme při spuštění dfs v \(G^T\) na \(B\) neměli navštívené ani \(A\), ani \(B\).
        Pak jsme z \(B\) došli do \(A\) (pozor - graf je invertovaný), a tedy jsme \(A\) opuštěli dříve, tedy bude v zásobníku dříve. Pokud bylo \(A\)
        již navštívené, už v zásobníku je. (V \(G^T\) nejde z \(A\) dojít do \(B\), protože pak by cesta byla obousměrná, což porušuje náš původní předpoklad.)

        Když je tedy \(B\) v zásobníku po \(A\), bude na \(B\) spuštěna konstrukce komponent (dfs v originálním grafu) dříve. A vzhledem k tomu, že z \(B\) nejde
        v \(G\) dojít do \(A\), tak \(B\) bude v komponentě bez \(A\). A při spuštění dfs na \(A\) už bude \(B\) navštívené, takže nebudou ve stejné komponentě.
        \item \(A \leftarrow B\) Toto je ekvivalentní \(B \rightarrow A\), což je popsáno v předchozím bodě.
        \item \(A \leftrightarrow B\) Nyní sice o pořadí v zásobníku nemůžeme nic říct, ale podívejme se na konstrukci komponent. Buď spustíme dfs nejdřív na \(A\),
        v tom případě dojdeme z \(A\) do \(B\) a oba vrcholy bude ve stejné komponentě. Nebo spustíme dfs na \(B\), v tom případě dojdeme do \(A\) a vrcholy budou
        také ve stejné komponentě.
    \end{itemize}
\end{enumerate}
Čímž jsme dokázali korekntonst hledání komponent.

\subsection{Spočítaní dosažizelnosti mezi komponentami}
Základní myšlenkou zde je sloučit si vrcholy v jedné komponentě na jeden (v rámci komponenty jde dojít z každého do každého).

Jak takový graf vytvoříme? Vytvoříme si vrcholů, kdy každý z nich repreznetuje jednu komponentu. (Budu jim říkat vrcholo-komponenty.) Poté pro každou hranu v originálním grafu \(G\):
\begin{itemize}
    \item Pokud vede v rámci jedné komponenty, tak ji ignorujeme. (Koncové vrcholy jsou ve stejných komponentách.)
    \item V opačném případě vytvoříme hranu mezi vrcholo-komponentami, které tato hrana spojuje. (Zachováváme původní orientaci hrany.)
\end{itemize}
Můžeme si rozmyslet, že pokud vede hrana mezi dvěma vrcholy, jde se z každého vrcholu v první komponentě dostat do počátečního vrcholu v hraně
(jsou ve stejné komponentě), poté jde po hraně přejít, a nakonec dojít do jakéhokoliv vrcholu v druhé komponentě.

No a nyní, když máme graf vrcholo-komponent, můžeme z každé z nich spustit dfs a uložit si, do kterých vrcholo-komponent jsme se dostali.
(Aby se v tom dobře hledalo, můžeme tohle ukládat třeba v tabulce boolů - jsme se schopni z komponenty v řádku dostat do komponenty ve sloupci?)

\subsection{Odpovědění dotazů}
Když načteme dotaz, jen se podíváme ve kterých komponentách jsou oba vrcholy, a poté se podíváme, jesli jde dojít z první komponenty do druhé.
Podle toho odpovíme.

\section{Časová složitost}
(Tak mimohodem značím \(N\) počet vrcholů a \(M\) počet hran, protože síla zvyku.)
\begin{enumerate}
    \item Najití silně souvislých komponent:
    \begin{itemize}
        \item Výroba inverze grafu. (Při dobré reprezentaci tohle jsme schopni každou hranu jednou projít a obrátit ji.) \(O(N+M)\)
        \item Dfs na \(G^T\): Každý vrchol navštívíme jednou, po každé hraně přejdeme dvakrát. \(O(N+M)\)
        \item Dfs na \(G\) - znovu \(O(N+M)\)
    \end{itemize}
    Celkem \(O(N+M)\).
    \item Spočítaní dosažizelnosti mezi komponentami (\(K\) - počet komponent):
    \begin{itemize}
        \item Výroba grafu vrcholo-komponent. \(O(K+M)\) (Každou hranu musíme projít.)
        \item Dfs na grafu: \(O(K (K+K^2)) = O(K^3))\) (Pouštíme \(K\) dfs, každé z nich projde všechny vrcholy i hrany.
        Hran je nejvýše kvadraticky tolik, co vrcholů.)
    \end{itemize}
    Celkem \(K^3\).
    \item Zodpovězení všech dotzů \(O(O)\)
\end{enumerate}
Celkem \(O(N+M+K^3+O)\), ale vzhledem k tomu, že v zadání je slíbeno, že počet silně souvislých komponent je nejvýše \(3\), tak
\(O(N+M+O)\).

\section{Časová složitost}
Pamatujeme si různé seznamy - komponenty pro vrcholy, navštívenost vrcholů, invertovaný graf, \dots
Největší z toho je invertovaný graf (\(O(N+M)\)) a matice dosažitelnosti \(O(K^2)\), takže celkem \(O(N+M+K^2) = O(N+M)\).

\end{document}