\documentclass{article}
\usepackage[utf8]{inputenc}
\usepackage{amsmath}
\usepackage{amssymb}
\usepackage{graphicx}
\usepackage[czech]{babel}
\usepackage{hyperref}
\hypersetup{
    colorlinks=true,
    linkcolor=blue,
    filecolor=magenta,      
    urlcolor=cyan,
    pdftitle={Overleaf Example},
    pdfpagemode=FullScreen,
    }
\graphicspath{ {./images/} }

\title{Bezpečnostní kamery}
\author{Daniel Skýpala}
\date{13. dubna 2022}

\begin{document}
\maketitle

\section*{Zamyšlení}
Začněme tím, že celá situace odpovídá grafu: Kamery jsou vrcholy a kabely jeho hrany. Pro tento grafu hledáme obarvení (přiřazení frekvencí = barev),
tak aby dva vrcholy spojené hranou neměli stejnou barvu.

Nejdřív si vezmeme nějaký vrchol \(i\), a podívejme se, s jakými dalšími vrcholy je spojen. Na to si rozepišme vrchol \(i\) jako jeho
prvočíselný rozklad:
\[i = 1 \cdot p_1 \cdot p_2 \cdots p_n\]
Kde platí, že \(p_i\) je prvočíslo a také \(1 \leq p_1 \leq p_2 \leq \cdots \leq p_n\). Prvočíselný rozklad budu nyní značit (pozor,
tohle je multimnožina, a pokaždé když dále zmíním podmnožinu, tak ve skutečnosti záleží i na počtu prvků):
\[P(i) = \{1, p_1, p_2, \dots, p_n\}\]

Nyní si rozmysleme, co platí, když vede hrana z \(i\) do \(j\).
\[ki = j\]
Rozložíme:
\[(1 \cdot p_{k1} \cdots p_{kn}) (1 \cdot p_{i1} \cdots p_{in}) = (1 \cdot p_{j1} \cdots p_{jn})\]
A to sloučíme (pozor, jednička je vlevo dvakrát):
\[P(k) \cup P(i) \setminus \{1\} = P(j)\]
\[P(i) \subseteq P(j)\]

Budeme potřebovat si ale dokázat ještě jednu věc. Konkrétně to, že když máme dva stejně dlouhé rozklady a jeden je podmnožinou druhého,
tak jde o rozklady stejného čísla (a opačně):
\[P(i) \subseteq P(j) \;\land\; |P(i)| = |P(j)| \; \Leftrightarrow \; i = j\]
Tak začneme s tím, že když \(i\) je rovno \(j\), tak jejich rozklady jsou jistě stejné a mají i stejnou velikost. 
Zbývá nám ještě implikace zprava doleva. Představme si, že máme dva stejně velké rozklady \(P(i), P(j)\) a zároveň víme, že \(P(i)\) je
podmnožinou \(P(j)\). \(P(j)\) musí tedy obsahovat všechny prvky z \(P(i)\), ale na žádný další nezbývá místo (jinak by nebyly stejně dlouhé).
Takže \(P(i) = P(j)\), ale protože součin rozkladu je roven číslu, ze kterého jsme ho vytvořili (\(i = \prod_{p \in P(i)} p\)), tak musí platit
i \(i = j\).

Nyní máme všechno, co potřebujeme:

\section*{Část 1}
Jak přiřazovat vrcholům barvy? Určitě vrchol 1 bude mít unikátní barvu (např. 1), protože je spojena se všemi zařízeními. Dvojka
bude mít tedy barvu 2. Čtyřku dělí jedna i dva, tedy nejnižší volná barva je 3 a tak dále.

Můžeme si všimnout, že s každou mocninou dvojky se délka rozkladu o jedna prodlouží, takže se nabízí přidělit barvu podle délky rozkladu:
\[f(i) = |P(i)|\]
Proč nyní nemáme dva různé vrcholy \(i, j\) spojené hranou se stejnou barvou? Uvažujme, že jsou spojené hranou \(P(i) \subseteq P(j)\)
a mají stejnou barvu \(|P(i)| = |P(j)|\). Ale jak jsme výše ukázali, to nastane právě tehdy, když \(i = j\), tedy zařízení nejsou různá.
Tedy žádné dva vrcholy spojené hranou nemají stejnou barvu.

A proč je počet barev optimální? Nejdříve se zeptejme, jaká je nejvyšší zvolená barva. To je velikost rozkladu čísla \(i < N\), kde
\(i\) má nejdelší rozklad. Čísel s nejdelším rozkladem, může být více, ale rozmysleme si, že když máme nějaké, můžeme provést následující věc:
Všechna prvočísla (ne jedničku) v rozkladu nahradíme dvojkou. Tím jsme jednak nezměnili barvu (protože délka rozkladu je stejná) a zároveň jsme se nedostali
nad \(N\), protože prvočísla jsme nahradili za dvojky (které nejsou větší). (A když jsou menší činitelé, je i menší součin.) Tedy největší barvu
v grafu bude mít mocnina dvojky. A o té víme že:
\[P(2^k) = \{1, 2_1, 2_2, \dots, 2_k\}\]
\[|P(2^k)| = k + 1\]
Pro velikost grafu \(N\) bude maximální počet barev tedy \(\lfloor \log_2 N \rfloor + 1\). A proč je tento počet barev optimální? Všimněme
si, že mocniny dvojky (včetně jedničky) tvoří úplný podgraf (každou mocninu dělí všechny nižší: \(\frac{2^a}{2^b} = 2^{a-b}\), což je zřejmě
celé číslo pro \(a > b\)), který má velikost \(\lfloor \log_2 N \rfloor + 1\) (mocnin dvojky je \(\log_2 N\) a k tomu ještě jednička).
A protože každá barva v úplném podgrafu může být jen jednou (jinak by vrcholy se stejnou barvou byly spojené hranou), tak určitě nemůžeme mít
méně jak \(\lfloor \log_2 N \rfloor + 1\), tedy naše obarvení je optimální.

Algoritmus, který použijeme pouze zjistí, kolik má daný identifikátor prvočíselných dělitelů a podle toho vrátí barvu.
To se dělá zkoušením čísel od \(2\) do \(i\), a když dané číslo dělí identifikátor, tak ho vydělíme a připočteme jedna k barvě.
(Více to rozebírat nebudu, protože se jedná o velmi známý \href{https://www.geeksforgeeks.org/prime-factor/}{algoritmus}, a
kdybychom chtěli být matematici, tak se ho můžeme snažit i \href{https://en.wikipedia.org/wiki/Integer_factorization}{zrychlit}.)
\begin{verbatim}
kamera.frekvence = 1
id = kamera.identifikátor
dělitel = 2
dokud (id != 1):
    když id mod dělitel == 0:
        id /= dělitel
        kamera.frekvence++
    jinak:
        dělitel++
\end{verbatim}
Časová složitost je v tomto případě \(O(id) = O(N)\). (Můžeme si snadno rozmyslet, že v každe iteraci cyklu buď klesne se zvýší dělitel o 1 - a
ten může být větší než id jen v poslední iteraci cyklu, nebo se id zmenší alespoň na půl.) Paměťová složitost je jen z pohledu na program \(O(1)\).

(Pozor - algoritmus je lineární k číslu N, ale exponenciální vzhledem k jeho binárního zápisu. - Takže se ve skutečnosti jedná o exponenciální algoritmus,
který je děsně pomalý, jakmile nám naroste N.)
\section*{Část druhá}
Pro řešení druhé části využijeme následujícího tvrení:
\[\forall i \in \mathbb{N} \land i > 1: \exists j \in \mathbb{N} \land i \not = j \land P(j) \subset P(i) \land |P(j)| + 1 = |P(i)|\]
Neboli pro každý vrchol \(i > 1\) existuje jiný vrchol \(j\) takový, že mezi \(i\) a \(j\) vede hrana a \(j\) má o jedna kratší rozklad
(tedy i o jedna menší frekvenci) než \(i\). Proč tohle platí? Představme si rozklad čísla \(i\) a vyškrtněme z něj prvočíslo. Tím
jsme dostali o jedna kratší rozklad, který je zároveň podmnožinou původního. A tento rozklad reprezentuje číslo \(j\).

Na obarvování půjdeme tak, že budeme obarvovat stejně jako v první části, akorát to musíme udělat z jiných informací. Všimneme si, že všechny
vrcholy (budu jim říkat předchůdci), ze kterých vede kabel do aktuálního vrcholu, mají menší barvu (Podmnožina je menší než multimnožina).
Pokud v \(i\)-té (\(i \in \mathbb{N}_0\)) sekundě budeme obarvovat barvou \(i+1\), tak určitě když dojde řada na aktuální vrchol, tak všichni jeho předchůdci budou v tu
chvíli obarvení. Zároveň všichni jeho předchůdci budou obarvení právě o sekundu dříve, protože jeden z nich má o jedna menší rozklad, tedy i barvu
a bude tedy obarven v \(i-1\). sekundě.

Z toho se nabízí následující algoritmus - v každé sekundě si v kabelech budeme posílat bit, jestli je vrchol, ze kterého vede kabel už obarvený.
V tu chvíli, kdy zjistíme, že všichni předchůdci už obarvení jsou, nastavíme současnému vrcholu barvu čas + 1, což nastane o 1 sekundu potom, než obarvíme
posledního z předchůdců (kvůli zpoždění v přenosu kabelů).

Důkaz o tom, že barva \(i\)-tého vrcholu je stejná jako v prvním případě, tedy \(f(i) = |P(i)|\), provedeme indukcí:
\begin{enumerate}
    \item Vrcholu 1 to přiřadí barvu \(|P(1)| = 1\). V nulté sekundě spustíme program a vrchol 1 je jediný, který nemá vstupní kabely.
    (Ostatní vrcholy jsou děleny beze zbytku jedničkou.) To znamená, že ve všech příchozích kabelech bude jednička (a co, že žádný takový není?).
    Tedy v čase 0 přiřadíme vrcholu 1 barvu \(t + 1 = 0 + 1 = 1\).
    \item Pokud přiřadíme správnou barvu všem vrcholům s menším číslem než současnému vrcholu \(i\), tak přiřadíme správně barvu i vrcholu \(i\).
    Jak jsem zmínil, všichni předchůdci vrcholu \(i\) jsou menší a mají i menší barvu než \(i\) (protože podmnožina je menší a musí mít i menší součin
    o ty prvky, které neobsahuje). 
    
    Pokud má vrchol \(i\) mít barvu \(|P(i)|\), tak (jak jsme dokázali) existuje jeho správně obarvený předchůdce \(j\), který má barvu \(|P(i)| - 1\).
    Tedy nejnižší čas, kdy lze obarvit vrchol \(i\), je kvůli zpoždění kabelů o jedna větší než čas, kdy jsme obarvovali vrchol \(j\). Zároveň, protože
    všichni předchůdci mají menší barvu, tak v tu chvíli, kdy je obarvený vrchol \(j\) jsou taky obarvení (protože jinak by neměli barvu podle předpokladu).
    Náš vrchol tedy bude obarven v čase o 1 později než vrchol \(j\):
    \[f(i) = t_i + 1 = t_j + 2\]
    A protože barva vrcholu \(j\) je určena podle času:
    \[f(j) = t_j + 1\]
    A vrchol \(j\) je obarvený správně a má o jedna menší rozklad než \(i\):
    \[|P(i)| - 1 = t_j + 1\]
    Z toho:
    \[f(i) = |P(i)|\]
    Takže jsme obarvili vrchol správně.
\end{enumerate}

Vzhledem k tomu, že obarvujeme stejně, jako v části jedna, tak správnost a optimálnost obarvení jsme již dokázali. Zbývá rozmyslet si čas -
Použijeme to, že nejvyšší barva v grafu je \(\lfloor \log_2 N \rfloor + 1\), a ta byla obarvena v přesně daný čas.
\[f(i) = t + 1\]
\[\lfloor \log_2 N \rfloor + 1 = t + 1\]
\[t = \lfloor \log_2 N \rfloor\]
To znamená, že po \(\lfloor \log_2 N \rfloor + 1\) sekundách (pozor - nultá sekunda se taky počítá), budou všechny barvy nastavené. Pokud bychom to chtěli
uvnitř kamer rozpoznat, tak můžeme zkontrolovat jen čas a v \(\lfloor \log_2 N \rfloor + 1\). sekundě skončit.

Jeden tik bude vypadat následovně:
\begin{verbatim}
když kamera.čas == 0: // čtení ze synchronizových hodin
    když kamera.p == 0: // počet vstupních zařízení
        kamera.připravena = 1
        kamera.frekvence = 1
    jinak:
        kamera.připravena = 0
jinak když kamera.připravena == 0:
    součet = 0
    pro každou zprávu v kamera.zprávy: // seznam všech přijatých zpráv
        součet += zpráva
    když součet == kamera.p:
        kamera.připravena = 1
        kamera.frekvence = kamera.čas + 1

pro každý kabel v kamera.výstupní_kabely: // seznam všech výstupních kabelů
    kabel.pošli(kamera.připravena)
\end{verbatim}
Časová složitost jednoho tiku je podle délky cyklů \(O(p + q) = O(N)\). Tiků je celkem logaritmicky mnoho, celkem tedy pro přenosovou rychlost
\(P\): \(O(N \log N + P\log N)\). Paměťová složitost, protože příchozí a odchozí informace nezapočítáváme, je \(O(1)\).

Ještě se nabízí zmínit, že bychom to stihli zvládnout v \(\lfloor \log_2 N \rfloor\) sekundách s menším trikem. - V nulté sekundě jsme schopni rozeznat vrchol 1 s barvou 1
(má 0 vstupních kabelů) a prvočísla s barvou 2 (1 vstupní kabel z jedničky) a oba typy obarvovat v 0. sekundě. Každý další vrchol by barvu akorát nastavoval
podle \(t+2\), protože jsme první dvě sekundy smrskli do jedné.
\end{document}